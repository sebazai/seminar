\documentclass[conference]{IEEEtran}
\IEEEoverridecommandlockouts

\usepackage{cite}
\usepackage{amsmath,amssymb,amsfonts}
\usepackage{algorithmic}
\usepackage{graphicx}
\usepackage{textcomp}
\usepackage{xcolor}
\def\BibTeX{{\rm B\kern-.05em{\sc i\kern-.025em b}\kern-.08em
    T\kern-.1667em\lower.7ex\hbox{E}\kern-.125emX}}

\begin{document}

\title{Decode the Work Spouse Code: Investigating the Desired Pair's Role and Pair Programming's Impact on Software Development}


\author{\IEEEauthorblockN{1\textsuperscript{st} Sebastian Sergelius}
\IEEEauthorblockA{\textit{Department of Computer Science} \\
\textit{University of Helsinki}\\
Helsinki, Finland \\
sebastian.sergelius@helsinki.fi}
}

\maketitle

\begin{abstract}
This document is a work in progress conference paper.
\end{abstract}

\begin{IEEEkeywords}
pair programming, collaboration, human factors
\end{IEEEkeywords}

\section{Introduction}

Pair programming has attracted attention in both industry and academia. Pair programming sessions are common in the agile methodology Extreme Programming where two programmers work on the same computer aiming to accomplish a software development feature or design \cite{10.5555/1076267}. In a pair programming session, one acts as an observer and the other as the driver. The driver operates the computer, while the observer tries to pinpoint possible defects, think of alternatives, guiding the partner with additional resources or think of the strategical significance of the implementation. Additionally, the roles should be swapped occasionally and the communication between the partners should stay interactive. Pair programming has gained interest within the software development industry for its potential to improve software quality, increase productivity, and encourage knowledge sharing, but its practical implementation presents challenges \cite{10.1145/2652524.2652529, Williams2000Strengthening}.

Microsoft researches did a survey to find out the various perceived benefits of pair programming in an industry setting \cite{10.1145/1414004.1414026}. According to the respondents, pair programming reduces code defects, enhances knowledge sharing, and improves the morale and creativity. Problems that the respondents reported were cost efficiency, scheduling, and traits related to individuals, such as personality, skill level or style differences in coding. 

Even though pair programming has its advantages, the effectiveness of pair programming is a topic of debate in the academia showing mixed results \cite{Hannay2009effectiveness}. Given the growing adoption of agile development methodologies in the software development industry, understanding pair programming benefits and challenges is crucial for knowing when, why and how to use this technique. Since pair programming is an interaction between two persons, human factors can cause conflicts or improve the perceived productivity. In addition, understanding how pair programming affect software engineering quality attributes, and which quality attributes have been addressed in academia could pinpoint future research directions.

This study provides a review of empirical studies on pair programming, focusing on perceived benefits and challenges between the pair, and the impact of pair programming on the software quality. The main goal of this study is to identify the human factors affecting the collaboration between the pair and how pair programming affects the programming task in comparison to programming the task solo.

Section 2 motivates the research topic and discusses the attributes considered in this article. The next section (Section 3) describes the method used, how literature was searched, and the research questions in more detail. The result section unveils the characteristics of a desirable partner and how pair programming affects the effectiveness of the programming task. Section 4 discusses the results, implications and limitations. We end with a conclusion wrapping up the when, how and why one should consider pair programming.

\section{Background and motivation}

As pair programming has gained popularity both in industry and academia, there has to be reasons to why pair programming is practiced. Assuming that assigning two persons to a task instead of one, pair programming has to bring some benefits, as otherwise it would be seen as waste of valuable resources. An answer to the when, how and why applying pair programming is beneficial, and how it affects the cost in comparison to an individual programmer doing the programming task strengthens the use case of pair programming.

A pair programming session is defined as two persons working on the same programming task with the goal to finish it according to the specifications. One can think of multiple factors that might affect how smoothly the session goes, and how they affect the pair programming session. The prominent aspect therefore is, what skills and traits are perceived as beneficial from a pair programming partner? In comparison to working solo, human factors in pair programming play a significant role on the effectiveness of a pair programming session. Therefore, identifying human factors affecting the pair programming session, and to what extent they have been studied would answer when it is beneficial to form a couple to do pair programming.

Even if a programmer knows when it would be beneficial to work in a pair on a task, a product owner in an agile development team might question its advantages. Therefore, identifying how effective pair programming is crucial. The cost can be measured by the duration and effort put into the task. Duration is measured from the start of the task and to the finish. Related to duration is effort, which is a measurement of pairs combined duration. Accuracy is measured by the correctness of the solution. In pair programming studies this can be measured with test cases or a thorough qualitative analysis of the code to find out if there are major defects present.

\section{Methods}

A descriptive literature review is conducted to provide an overview of existing scientific literature on pair programming based on the pair programming papers in the Empirical Software Engineering and Measurement conference. This research represents a literature review on the perceived effects of pair programming on the quality of the task in progress, and what characteristics are desirable from a pair programming partner. 

\subsection{Articles selected}

As the topic is related to empirical evidence on a software practicing technique the reputable ESEM conference is used as the main source for literature. In the ACM Digital Library the ESEM filter is applied and a search string of "pair programming" present in author keywords. A total of four articles were found that were available on the ACM, see \cite{10.1145/2652524.2652529, 10.1145/1414004.1414026, 10.1145/1852786.1852816, 10.1145/1159733.1159749}.

Additionally, each four paper's abstract is used as a search string using the free version of Keenious. The highest cited paper from Keenious is picked. If Keenious search provided a paper already picked on another abstract search, the second highest cited paper is picked. These paper have an inclusion criteria of the abstract mentioning "pair programming". Therefore, we end up with 3 papers from Keenious, see \cite{Williams2000Strengthening, Arisholm2007Evaluating, Hannay2009effectiveness}. However, a closer look indicates that \cite{Williams2000Strengthening, Arisholm2007Evaluating} found by Keenious would have been found by snowballing the ACM articles, and \cite{Hannay2009effectiveness} with backwards snowballing. Moreover, the excluded paper \cite{ChamorroPremuzic2003Personality} found by Keenious was in fact a reference of \cite{10.1145/1852786.1852816}.

Each paper is thematically categorized based on if either their title or abstract mentions human factors or the quality characteristics of software development. A paper can be in multiple categories. Based on the title and abstracts the papers are categorized in the following way: papers \cite{10.1145/2652524.2652529, Williams2000Strengthening, 10.1145/1414004.1414026, Arisholm2007Evaluating, 10.1145/1852786.1852816, Hannay2009effectiveness} mentions human factors, and papers \cite{Williams2000Strengthening, 10.1145/1414004.1414026, Arisholm2007Evaluating, 10.1145/1159733.1159749, Hannay2009effectiveness} discusses software quality aspects. The snowballing method for found articles will be applied for additional relevant literature.

\subsection{Research Questions}

Based on the seven found articles and their thematic categorization, this paper seeks to answer the following two research questions.

\begin{itemize}
    \item \textbf{RQ1.} What are the perceived skills and traits of an effective programming partner?
    \item \textbf{RQ2.} How does pair programming affect the cost and correctness of the programming task?
\end{itemize}

To answer RQ1 in detail, each of the papers will be evaluated by finding common themes related to human traits, and to evaluate whether found papers discuss these characteristics on how they impact the success of the pair programming session. This includes aspects such as communication skills, personality types, and expertise levels. The goal is to understand how these human factors influence the productivity and overall success of a pair programming session.

From papers related to the RQ2, trying to find the quality attributes affecting the software that are influenced by pair programming. Whether pair programming leads to higher quality code, whether it improves the time-to-market, and how it compares to individual programming in terms of producing code that meets the requirements. 

\section{Results}

Pair programming has been studied both in industry and academia \cite{Williams2000Strengthening, 10.1145/1414004.1414026, Hannay2009effectiveness}. There are many aspects that could affect the results of a pair programming session. Some affecting attributes are personality, skill level, complexity of task and the time the pair has worked together before (pair jelling). Furthermore, pair programming can be helpful to reduce the duration to finish the task, to reduce bugs in the code, and to improve the design of the program \cite{10.1145/2652524.2652529}. In addition, knowledge sharing, focus on the task, improved learning, and to building trust within the team are seen as benefits of pair programming.

\subsection{Perceived skills and traits of an effective programming partner}

Pair programming requires more than just software development skills. An essential skill is communication between the pair. Developers need to be able to provide clear instructions to the pair in case the other pair does not necessarily understand the task goal \cite{10.1145/2652524.2652529}. Additionally, the receiver of this new knowledge might ask questions that are not related to the task, which requires the responder to focus on the task in progress and possibly deviate such questions or answer them for later once the task is finished \cite{10.1145/2652524.2652529}. 

Microsoft researchers surveyed people who has worked in pairs and analyzed the answers based on what characteristics the partner looks for from their pair \cite{10.1145/1414004.1414026}. According to the respondents, listed from most to least desirable, five traits of a good programming partner are complementary skills, flexibility, good communications, smart and personable. A desirable partner is one who complements their thinking and skills, both technically and in design. Flexibility requires the partner to be open-minded and to acknowledge that there are multiple ways to achieve similar results. A good listening, verbal and argumentation skills are desirable when communicating. Additionally, a partner with an equivalent or better skill level that is able to focus on the task are within the top 10 traits of a good partner as well.

A great deal of previous research into pair programming has focused on the duration it takes to accomplish a task \cite{10.1145/2652524.2652529, Williams2000Strengthening, Hannay2009effectiveness}. Williams et al. studied pair programming in academia and showed that working as a pair the task was completed 40--50\% faster \cite{Williams2000Strengthening}. Additionally, each of the pairs handed in the assignment on time, whereas individuals were late or did not hand in the task at all, indicating that peer pressure impacts the focus and determination. Moreover, finishing the task boosts the morale and social cohesion of the pair. Working together for a longer period also brings up the strengths and weaknesses of the individuals, which helps pairs assign particular parts of the task or determine which role suits better for each individual.

Pair programmers stated that difficulties arise if the other person has a too big or small ego \cite{Williams2000Strengthening}. With a big ego, the pair does not necessarily have the flexibility to understand that there are multiple ways to accomplish the task, and a small ego was perceived as having trouble asserting themselves and therefore their contribution is low. Similarly, the questionnaire by Respondents in the Microsoft survey pointed out that problems occur when there are personality clashes, disagreements, programming style differences or skill differences between the pair \cite{10.1145/1414004.1414026}.


% Possible answers from \cite{Williams2000Strengthening, 10.1145/1414004.1414026, Arisholm2007Evaluating, 10.1145/1852786.1852816, Hannay2009effectiveness} 

\subsection{The cost and correctness of practicing pair programming}

Previous studies have explored the relationships between pair programming and the time needed to finish the task \cite{ 10.1145/1414004.1414026, 10.5555/377517.377531, Arisholm2007Evaluating}. The results show that the time needed to finish the task is reduced. However, the cost of pair programming a task is higher compared to individuals, as there are two persons working on the same task. Nevertheless, with additional manpower the correctness of the solution is increased \cite{Arisholm2007Evaluating, Williams2000Strengthening}. 

Williams et al. research showed that the used time to complete a programming task dropped by 40--50\% in comparison to individuals \cite{Williams2000Strengthening}. In the same study, they also compared the correctness of the solution by running test cases. All individuals managed to pass under 80\% of the test cases while the pairs managed to pass over 80\% of the test cases, even one pair getting up to 94.4\% coverage. 

Another study conducted by Arisholm et al. compared how expertise and system complexity affected the duration, effort and accuracy of pair programming compared to individuals \cite{Arisholm2007Evaluating}. Their result show that there was no statistical significance on duration between pairs and individuals. However, when taking into account the system complexity, easier tasks were programmed significantly faster than complex tasks. Additionally, the most significant decrease in duration was seen by intermediate developers on the simple task with a 39\% decrease in duration. Moreover, compared to individuals the effort required by the intermediate developers for the simple task increased only by 22\%. However, the correctness in the same setting only increased by 4\%. The statistically significant result can be seen by combining junior software developer pairs on a complex tasks to increase the correctness.

- Their study showed that pair programmers in the uni experiment increased the test coverage compared to individual programmers, the longer a defect is in a system, the more costly it is to repair \cite{Williams2000Strengthening}



- Juniors had help of PP when a task is a simple maintanance task \cite{Arisholm2007Evaluating}

% - Go into more detail of these papers that are thematically categorized summarizing.
% \cite{Williams2000Strengthening, 10.1145/1414004.1414026, Arisholm2007Evaluating, 10.1145/1159733.1159749, Hannay2009effectiveness} 


\section{Discussion}

Pair programming research show that the area of research is enormous, as there are a magnitude of affecting variables.


\subsection{RQs}

- Difference between experiments in academia and industry???

- Discussion if there are any tips and tricks for pair programming to think about when practicing pair programming

- Constant code review happening

\subsection{Limitations}

The research method had a major limitation when it comes to finding relevant literature. Even though the main goal was to identify the perceived characteristics of a desirable pair programming partner, searching relevant literature based solely on the ESEM conference definitely increased the odds of missing some relevant literature. In addition, selecting the most cited papers from Keenious might drop out relevant literature as some papers get more citations based on the writer or the based on the type of paper. Furthermore, citation count grow by time, therefore most cited papers might be old and thus there could be more comprehensive studies for the topic out there.

\section{Conclusion}

Wrap it up. Max 1 column. So what? 

- Different studies have different results. 

- Pair programming is beneficial if you do not care about the cost. Decreases the number of defects and enhances knowledge sharing. 

- Pair programming might work for you and your colleague, give it time

- When trying out pair programming, be open minded to different views on how things can be done, be prepared to think out aloud when you are the driver




\bibliographystyle{ieeetr}
\bibliography{refs}

\end{document}
