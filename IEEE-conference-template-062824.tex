\documentclass[conference]{IEEEtran}
\IEEEoverridecommandlockouts

\usepackage{cite}
\usepackage{amsmath,amssymb,amsfonts}
\usepackage{algorithmic}
\usepackage{graphicx}
\usepackage{textcomp}
\usepackage{xcolor}
\def\BibTeX{{\rm B\kern-.05em{\sc i\kern-.025em b}\kern-.08em
    T\kern-.1667em\lower.7ex\hbox{E}\kern-.125emX}}

\begin{document}

\title{Decode the Work Spouse Code: Investigating the Desired Pair's Role and Pair Programming's Impact on Software Development}


\author{\IEEEauthorblockN{1\textsuperscript{st} Sebastian Sergelius}
\IEEEauthorblockA{\textit{Department of Computer Science} \\
\textit{University of Helsinki}\\
Helsinki, Finland \\
sebastian.sergelius@helsinki.fi}
}

\maketitle

\begin{abstract}
\textit{Context:} Pair programming is often perceived of boosting effectiveness and increasing the accuracy of the programming task.
\textit{Goal:} Identify human factors that influence the effectiveness and enjoyment of pair programming and how it affects the cost and correctness of the task. Understand when, how and why it is beneficial to do pair programming.
\textit{Method:} A descriptive literature review and categorizing papers if their title or abstract talks about human factors or quality attributes related to the programming task.
\textit{Results:} A programming partner is expected to have similar traits as your spouse: communicative, open-minded and complementary skills. Senior pairs does not have significant benefits on effort, duration or correctness. However, on complex software tasks it increased the correctness for juniors and intermediate developers. 
\textit{Conclusion:} The perceived benefits of pair programming include increased enjoyment of work, a shorter time-to-market, and improved correctness.
\end{abstract}

\begin{IEEEkeywords}
pair programming, collaboration, human factors
\end{IEEEkeywords}

\section{Introduction}

Pair programming has attracted attention in both industry and academia. Pair programming sessions are common in the agile methodology Extreme Programming where two programmers work on the same computer aiming to accomplish a software development feature or design \cite{10.5555/1076267}. In a pair programming session, one programmer acts as an observer and the other as the driver. The driver operates the computer, while the observer tries to pinpoint possible defects, think of alternatives, guide the partner with additional resources, or think of the strategic significance of the implementation. Additionally, the roles should be swapped occasionally and the communication between the partners should stay interactive. Pair programming has gained interest within the software development industry for its potential to improve software quality, increase productivity, and encourage knowledge sharing, but its practical implementation presents challenges \cite{10.1145/2652524.2652529, Williams2000Strengthening}.

Since pair programming is an interaction between two persons, human factors can cause conflicts or improve perceived productivity. Microsoft researchers surveyed their programmers to find out the various perceived benefits of pair programming in an industry setting \cite{10.1145/1414004.1414026}. According to the respondents, programming in pairs reduces code defects, enhances knowledge sharing, and improves morale and creativity. Problems that the respondents reported were cost efficiency, scheduling, and traits related to individuals, such as personality, skill level, or style differences in programming. 

Even though pair programming has its advantages, the effectiveness is a topic of debate in academia showing mixed results \cite{Hannay2009effectiveness}. Given the growing adoption of agile development methodologies in the software development industry, understanding pair programming benefits and challenges is crucial for knowing when, how, and why to program in pairs. In addition, understanding how it affects software engineering quality attributes, and which quality attributes have been addressed in academia could pinpoint future research directions.

This study reviews empirical studies on pair programming, focusing on perceived benefits and challenges between the pair, and the impact of pair programming on the software quality. The main goal of this study is to identify the human factors affecting the collaboration between the pair and how pair programming affects the programming task in comparison to programming the task solo.

Section 2 motivates the research topic and discusses the attributes considered in this article. The next section (Section 3) describes the method used, how literature was searched, and the research questions in more detail. The result section unveils the characteristics of a desirable partner and how pair programming affects the effectiveness of the programming task. Section 4 discusses the results, implications, and limitations. We end with a conclusion wrapping up the when, how, and why one should consider pair programming.

\section{Background and motivation}

As pair programming has gained popularity both in industry and academia, there have to be reasons why pair programming is practiced. As in pair programming, two persons are assigned to a task instead of one, pair programming has to bring some benefits, as otherwise it would be seen as a waste of valuable resources. An answer to the when and why it would be beneficial to apply pair programming, from the perspective of how it affects the overall cost and quality of the software compared to an individual programmer doing the programming task, could strengthen the use cases of pair programming.

One can think of multiple factors that might affect how smoothly the session goes and how they affect the pair programming session. The prominent aspect therefore is, what skills and traits are perceived as beneficial from a pair programming partner in comparison to working solo, as human factors in pair programming play a significant role in the effectiveness of a pair programming session. Therefore, identifying human factors affecting the pair programming session, and to what extent they have been studied would answer when it is beneficial to form a couple to do pair programming.

Even if an agile team knows when it would be beneficial to work in a pair on a task, a product owner in a team might question its advantages. Therefore, identifying how effective pair programming is crucial. The effectiveness of pair programming will be considered by the cost which can be measured by the duration and effort put into the task. Duration is measured from the start of the task to the finish. Related to duration is effort, which is a measurement of pairs combined duration. Accuracy is measured by the correctness of the solution. Studies have measured accuracy with either test cases or a thorough qualitative analysis of the code to find out if there are major defects present.

\section{Methods}

A descriptive literature review is conducted to provide an overview of existing scientific literature on pair programming based on the pair programming papers at the Empirical Software Engineering and Measurement (ESEM) conference. This research represents a literature review on the perceived effects of pair programming on the quality of the task in progress, and what characteristics are desirable from a pair programming partner. 

\subsection{Articles selected}

As the topic is related to empirical evidence on a software practicing technique the reputable ESEM conference is used as the main source for literature. In the ACM Digital Library, the ESEM filter is applied, and a search string of "pair programming" is present in article keywords. A total of four articles were found that were available on the ACM, see \cite{10.1145/2652524.2652529, 10.1145/1414004.1414026, 10.1145/1852786.1852816, 10.1145/1159733.1159749}.

Additionally, each four paper's abstract is used as a search string using the free version of Keenious. The highest cited paper from Keenious is picked. If Keenious search provided a paper already picked on another abstract search, the second highest cited paper is picked. These papers have inclusion criteria of the abstract mentioning "pair programming". Therefore, we end up with 3 papers from Keenious, see \cite{Williams2000Strengthening, Arisholm2007Evaluating, Hannay2009effectiveness}. However, a closer look indicates that \cite{Williams2000Strengthening, Arisholm2007Evaluating} found by Keenious would have been found by snowballing the ACM articles, and \cite{Hannay2009effectiveness} with backward snowballing. Moreover, the excluded paper \cite{ChamorroPremuzic2003Personality} found by Keenious is a reference of \cite{10.1145/1852786.1852816}.

Each paper is thematically categorized based on whether either their title or abstract mentions human factors or the quality characteristics of software development. A paper can be in multiple categories. Based on the title and abstracts the papers are categorized in the following way: papers \cite{10.1145/2652524.2652529, Williams2000Strengthening, 10.1145/1414004.1414026, Arisholm2007Evaluating, 10.1145/1852786.1852816, Hannay2009effectiveness} mentions human factors, and papers \cite{Williams2000Strengthening, 10.1145/1414004.1414026, Arisholm2007Evaluating, 10.1145/1159733.1159749, Hannay2009effectiveness} discusses software quality aspects. The snowballing method for found articles will be applied to additional relevant literature.

\subsection{Research Questions}

Based on the seven found articles and their thematic categorization, this paper seeks to answer the following two research questions.

\begin{itemize}
    \item \textbf{RQ1.} How do human factors influence the effectiveness and enjoyment of a pair programming partnership?
    \item \textbf{RQ2.} How does pair programming affect the cost and correctness of the programming task?
\end{itemize}

To answer RQ1 in detail, each of the papers will be evaluated by finding common themes related to human traits, and to evaluate whether the found papers discuss these characteristics on how they impact the success of the pair programming session. This includes aspects such as communication skills, personality types, and expertise levels. The goal is to understand how these human factors influence the productivity and overall success of a pair programming session.

From papers related to RQ2, we are trying to find the quality attributes affecting the software that are influenced by pair programming. Whether pair programming leads to higher quality code, whether it improves the time-to-market, and how it compares to individual programming in terms of producing code that meets the requirements. 

\section{Results}

Pair programming has been studied both in industry and academia \cite{Williams2000Strengthening, 10.1145/1414004.1414026, Hannay2009effectiveness}. Many traits could affect the effectiveness of a pair programming session. Some affecting attributes are personality, skill level, the complexity of the task, and the time it takes to adjust to pair programming instead of programming solo (pair jelling). Furthermore, pair programming can be helpful to reduce the duration to complete the task, to reduce bugs in the code, and to improve the design of the program \cite{10.1145/2652524.2652529}. In addition, knowledge sharing, focus on the task, improved learning, and building trust within the team are perceived as benefits of pair programming.

\subsection{Human factors of a programming partnership}

Pair programming requires more than just software development skills. An essential skill is communication between the pair. Developers need to be able to provide clear instructions to the pair in case the other pair does not necessarily understand the task goal \cite{10.1145/2652524.2652529}. Additionally, the receiver of this new knowledge might ask questions that are not related to the task, which requires the responder to focus on the task in progress and possibly deviate from such questions or answer them later once the task is finished \cite{10.1145/2652524.2652529}. However, knowledge transfer is perceived as one of the beneficial factors when conducting pair programming \cite{10.1145/2652524.2652529}.

Microsoft researchers surveyed people who have worked in pairs and analyzed the answers based on what characteristics the partner looks for from their partner \cite{10.1145/1414004.1414026}. According to the respondents, the five most desirable traits of a good programming partner are complementary skills, flexibility, good communication, smart, and personable. A desirable partner complements their thinking and skills, both technically and in design. Flexibility requires the partner to be open-minded and to acknowledge that there are multiple ways to achieve similar results. Good listening, verbal, and argumentation skills are desirable when communicating. Additionally, a partner with an equivalent or better skill level that can focus on the task is within the top 10 traits of a desirable partner as well.

A great deal of previous research into pair programming has focused on the duration it takes to accomplish a task \cite{10.1145/2652524.2652529, Williams2000Strengthening, Hannay2009effectiveness}. Williams et al. studied pair programming in academia and showed that working as a pair the task was completed 40--50\% faster after the individuals had overcome pair jelling \cite{Williams2000Strengthening}. Similar results, that after pair jelling the duration to complete the task is reduced, have been reported by \cite{1541842}. Additionally, each of the pairs handed in the assignment on time, whereas individuals were late or did not hand in the task at all. Moreover, finishing the task boosts the morale and social cohesion of the pair. Working together for a longer period brings up the strengths and weaknesses of the individuals.

Pair programmers stated that difficulties arise if the other person has a too big or a small ego \cite{Williams2000Strengthening}. With a big ego, the pair does not necessarily have the flexibility to understand that there are multiple ways to accomplish the task, and a small ego was perceived as having trouble asserting themselves and therefore their contribution is low. Similarly, in the survey by Microsoft researchers, the respondents pointed out that problems occur when there are personality clashes, disagreements, programming style differences, or skill differences between the pair \cite{10.1145/1414004.1414026}. In addition, neuroticism personality traits in pair programming do not seem to have an impact on the pair programming session \cite{10.1145/1852786.1852816}.

With a suitable pair, programmers have reported that they enjoy working more in pairs and they feel more comfortable with their solution \cite{Williams2000Strengthening}. However, programmers stated that working on simpler task alone is more effective than in pairs. 


\subsection{The cost and correctness of practicing pair programming}

Previous studies have explored the relationships between pair programming and the time needed to finish the task \cite{10.1145/1414004.1414026, 10.5555/377517.377531, Arisholm2007Evaluating}. The duration is reduced and the correctness of the solution increases. However, the effort (cost) of pair programming a task is higher compared to individuals. Moreover, when taking into account task complexity and the expertise, the results vary \cite{Arisholm2007Evaluating}. Nevertheless, with additional manpower, the correctness of the solution is increased or stayed the same.

Williams et al. research showed that the used time to complete a programming task dropped by 40--50\% in comparison to individuals \cite{Williams2000Strengthening}. In the same study, they also compared the correctness of the solution by running test cases. All individuals managed to pass under 80\% of the test cases whereas the pairs managed to pass over 80\% of the test cases, even one pair getting up to 94.4\% coverage. In addition, compared to the traditional code and review process being separate, pair programming reduces the total development time on a simple task in a classroom setting \cite{10.1145/1159733.1159749}. On a more complex task in industry, the total effort was slightly higher. However, working in pairs significantly reduced the major problems in both settings. 

A statistical study compared how expertise and system complexity affected the duration, effort, and accuracy of pair programming compared to individuals \cite{Arisholm2007Evaluating}. Their results show that there was only an 8\% decrease in duration in favor of pairs. However, when taking into account the system complexity, pairs programmed the simple tasks significantly faster than complex tasks. The most significant decrease in duration was seen by pairs of intermediate developers on the simple task with a 39\% decrease in duration. In comparison to individuals, the effort required by the intermediate developer pairs for the simple task increased only by 22\%, while juniors with the same task complexity had a significant 109\% increase and seniors 55\%. In the complex task, an increase in correctness was seen by juniors (149\%) and intermediate (92\%) pairs. However, the increased correctness requires significantly more effort. The key takeaway is that simpler task had shorter duration, whereas complex tasks had increased correctness, however all came with a bigger cost \cite{Arisholm2007Evaluating}.

The effort invested in pair programming a complex task can reduce the overall cost. IBM reported that they spent 250 million US dollars on fixing 30,000 defects, which converts into 8,000 dollars per defect \cite{10.5555/377517.377531}. Therefore, taking into account quality assurance and other side expenses on defect fixing can reduce the overall cost of the project, as it approximately takes 4--16 hours per found defect for quality assurance engineers. An argument in favor of pair programming by \cite{10.5555/377517.377531}, where they argued that a programmer implements approximately 100 defects per 1000 lines of code (LOC). With a speed of 50 LOC/hour for a 50,000 LOC program, there will be 5000 defects. After code reviewing, approximately 70\% of the defects are found, therefore individuals would have 1500 defects left, whereas pair programmers would have 15\% less (1275) based on their earlier findings in \cite{Williams2000Strengthening}. If a defect takes approximately 8 hours to find and fix, it would take an extra 1800 hours for the quality assurance engineers to fix the program done by individuals. A 50,000 LOC program would take an individual 1000 hours to program, and naively assuming pairs do 50 LOC/hour and doubling the effort for a pair (2000 hours), this would still be in favor of pair programming. 

Pair programming has also been considered as an alternative for the traditional code review process \cite{10.1145/1159733.1159749}. The total development cost, consisting of development and quality assurance, was 4\% higher and the software had 39\% less defects in industry setting. For undergraduate students and pairs returning a similar quality task showed that the total development cost was 24\% less. Respondents in a survey said that the pair analysis and pair design was seen as more important than pair implementation \cite{Williams2000Strengthening}. 


\section{Discussion}

This literature review was conducted to try to find an answer on how human factors affect the effectiveness and enjoyment of a pair programming partnership. In addition, the study showed how pair programming affects the cost and correctness of the task in the making. 

A summary of the answers for our research questions:
\begin{itemize}
    \item \textbf{RQ1.} How do human factors influence the effectiveness and enjoyment of a pair programming partnership?
    

    People wish to have a communicative, open-minded and skillful partner with complementary skills. A big ego is seen as closed-minded, whereas a small ego can be seen as a person being afraid to communicate own thoughts. Personality clashes, disagreements and programming style differences, or skill differences causes problems.

    The shift from working solo to having a pair is cumbersome, but after a couple of pair sessions, pair programming improved efficiency and enjoyment of work. Focus and discipline for students can be seen by pairs always returning assignments in comparison to individuals.

    Overall, people enjoy pair programming more than working alone and feel more confident in their work.

    
    \item \textbf{RQ2.} How does pair programming affect the cost and correctness of the programming task?
\end{itemize}


Pair programming research shows that the area of research is enormous, as there is a magnitude of affecting variables. The results of the studies are mixed, as each study has taken into account different aspects. For example, pair jelling was taken into account in Williams et al. study \cite{Williams2000Strengthening}, whereas the study conducted by Arisholm et al. in the industry setting did not take into account pair jelling \cite{Arisholm2007Evaluating}.

When comparing the results of \cite{Arisholm2007Evaluating} and \cite{10.1145/1159733.1159749}, both showed that complex tasks benefit from pair programming, but it comes with a significantly higher effort, meaning that the cost is higher. However, the duration seemed similar to individuals. Therefore, if cost is not an issue and the time-to-market, then pair programming is beneficial for complex tasks. Results from a study combining pairs of different skill levels on a complex task would strengthen this insight. In addition, comparing pairs that have worked together and newly created pairs would have an effect on the outcome as Williams et al. showed that after a while, the pairs worked significantly faster after they got to know each other and understood how to work in pairs \cite{Williams2000Strengthening}.

Pair programming is beneficial for students, as this adds peer pressure and enhances the motivation to finish the task by deadline, as it does not affect only the individuals progress in studies as shown by \cite{Williams2000Strengthening}. However, similar peer pressure effect was not shown in the survey by Microsoft \cite{10.1145/1414004.1414026}. 


\subsection{Limitations}

The research method had a major limitation when it came to finding relevant literature. Even though the main goal was to identify the perceived characteristics of a desirable pair programming partner, searching relevant literature based solely on the ESEM conference increased the odds of missing some relevant literature. In addition, selecting the most cited papers from Keenious might drop out relevant literature as some papers get more citations based on the writer or the based on the type of paper. Furthermore, citation count grows over time, therefore most cited papers might be old and thus there could be more comprehensive studies for the topic under study.

A second limitation is that almost all of the pair programming experiments involve a small amount of pairs. Therefore, the generalization of wether pair programming is benefical is questionable.

\section{Conclusion}

To conclude, this study was a descriptive literature review of the pair programming papers published at the ESEM conference. The goal of this study was to find out what are the desired traits of a pair programming partner and how pair programming affects the cost and quality of the software. A desirable programming partner is an open-minded partner with great communication and complementary development skills. By conducting pair programming, knowledge of the software and technical skills are exchanged between the partners. Pair programming improves the time-to-market when developing complex tasks and it almost always leads to better results. Pair programming requires more effort in man-hours, therefore being more costly. However, the amount of defects is less, which can lead to significant cost savings when trying to repair these in the system. 




\bibliographystyle{ieeetr}
\bibliography{refs}

\end{document}
